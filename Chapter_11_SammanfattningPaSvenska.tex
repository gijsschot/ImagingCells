\chapter{Sammanfattning pa Svenska}

In this thesis we want to develop new methods to image biological diversity at the molecular level. Understanding this diversity is very important as it can play a major role in development of deceases. The thesis focuses on two topics: the imaging \textit{living} cells and the development of a computational tool to assess variability within a population of images.
\\\\
In molecular biology structure and function are closely related. This however does not mean that each object has one single static shape. On the contrary, many objects have multiple conformations and only through understanding the variation, their function becomes interpretable. A further complication arises from the fact that the variation is often highly dependent on the exact environment the objects are in. Outside of their natural environment, molecular machines might start operating differently. A tool that could image molecular machinery inside their natural environment, the living cell, is therefore required for us to truly understand the structure of molecular life.
\\\\
The amount of detail we can image in an object is limited to about the size of the wavelength of the light used. Molecular machines are build up out of atoms, so light with a wavelength similar to the size of an atom is needed to see single atoms and thus understand the functioning of molecular machines fully. This type of light is called X-ray radiation. Another fundamental principle is that the smaller an object is, the stronger light is needed in order to record enough information to record an image. Strong beams however have the disadvantage that they damage the objects themselves. This leads to the limitation that single objects smaller than 100 atoms in diameter can not be visualized using ordinary techniques.
\\\\
The development of a new type of X-ray laser called an x-ray free-electron lasers, that produces ultra short and extremely bright X-ray pulses might hold the solution to this limit imposed by sample damage. The power of the pulse ensures that even individual molecular machines can be imaged in atomic detail. These pulses do damage the sample, but because they are ultra-short the damage only happens after the pulse has passed. The recorded image is therefore an image of the undamaged object. This principle is called diffraction before destruction. 
\\\\
The word diffract indicates that we are not recording images of an object in the same way that a photo camera captures images. Instead we measure a so called diffraction pattern. The 2D diffraction pattern can be converted to a real image (2D) in a process called image reconstruction. The feasibility of diffraction before destruction and the image reconstruction process has been shown for a variety of non-biological and biological samples in 2D.  And recently the first 3D model of a biological object has been generated from many diffraction patterns generated using this method.
\\\\
In 2008, researchers showed that it is theoretically possible to image small living cells at sub-nanometer resolution in 2D. The main work of the thesis was an experimental verification of this prediction. This study showed that using an XFEL that was weak compared to the imaginary ones used in the predictions, it is possible to record images with signal up to 4 nm resolution which is equivalent to 40 atomic diameters. The problem of these images is that the strong signal caused the detector to saturate. We therefore had to resort to weaker diffraction patterns for the analysis. This posed the limit on the resolution of the recovered object. We have suggested improvements to avoid detector saturation, but these suggestions have not yet been tested experimentally.
\\\\
Due to the variability between different cells it is not trivial to fully automate the process of object reconstruction from the recorded diffraction, something that is needed to study the variability of cells in high detail, or the molecular machines inside of them. For the automation to succeed it necessary to predict the shape of the object from the recorded image itself. Furthermore, other parameters such as saturation or having two cells in the beam at once also have to automatically identified.
\\\\
To solve this problems I have developed a software suite called RedFlamingo. RedFlamingo can be used to assess the quality of individual images and deduces several essential features for image reconstruction. In some cases it can also determine whether the observed variety originate from a biological source, or whether it is a result from the experiment. It turns out that this feature can be useful in the process of deriving a 3D model from many 2D images. 
\\\\
I am very excited to be part of the development of this technique. Measuring images without saturation effects might open the door to observing the biological processes occurring inside living cells. For the future I see the study of variability in isolated molecular machines as the first milestone. After that I hope that these machines can be studied together with their natural environment, allowing us to visualize the organisation of life.
