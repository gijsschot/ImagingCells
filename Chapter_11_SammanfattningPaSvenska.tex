\begin{otherlanguage}{swedish}

\chapter{Sammanfattning på Svenska}
Denna avhandling behandlar utvecklling av nya metoder för avbildning av biologisk variation på molekylär nivå. Att förstå denna variation är mycket viktigt eftersom små ändringar i t.ex. proteiners struktur kan spela en viktig roll för uppkomsten av sjukdomar. Avhandlingen fokuserar på två ämnen: bildbehandling av levande celler och utveckling av beräkningsverktyg för att bedöma variationen inom en samling bilder.

Från ett molekylärbiologiskt perspektiv är struktur och funktion nära relaterade. Detta innebär dock inte att varje objekt har en enda statisk form. Tvärtom har många föremål flera konformationer och endast genom att förstå denna strukturvariation kan vi tolka deras funktion. En ytterligare komplikation härrör från det faktum att variationen ofta är mycket beroende av den exakta miljön objekten befinner sig i. Utanför sin naturliga miljö kan molekylära maskiner börja fungera annorlunda. Ett verktyg som kan avbilda molekylära maskiner i sin naturliga miljö, den levande cellen, är därför nödvändigt för att vi verkligen ska förstå livets struktur på molekylär nivå.

Den detaljnivå på vilken vi kan avbilda ett objekt är begränsad till ungefär storleken på våglängden för det ljus som används. Molekylära maskiner byggs upp av atomer, så ljus med en våglängd som ligger nära storleken på en atom behövs för att kunna se enskilda atomer och därmed förstå molekylära maskiners funktion. Denna typ av ljus kallas röntgenstrålning. En annan grundläggande princip är att ju mindre ett objekt är, desto starkare ljus behövs för att samla in tillräckligt med information för att framställa en bild. Stark strålning har dock nackdelen att den skadar föremålet som bestrålas. Detta medför att enskilda objekt mindre än 100 atomer i diameter inte kan avbildas med vanliga tekniker.

En ny typ av röntgenlaser som kallas röntgenfrielektronlaser, producerar ultrakorta och extremt intensiva röntgenpulser och kan vara lösningen till att undvika provskador. Pulsens styrka säkerställer att även enskilda molekylära maskiner kan avbildas i atomär detalj. Dessa pulser skadar provet, men eftersom de är extremt korta uppstår skadorna efter att pulsen har passerat provet. Bilden föresteller därför det oskadade objektet. Denna princip brukar kallas diffraktion före destruktion.

Ordet diffraktera indikerar att vi inte avbildar objekt på samma sätt som en fotokamera tar bilder. Istället mäter vi ett så kallat diffraktionsmönster som visar hur ljuset sprids och ändrar riktning när det träffar objektet. 2D-diffraktionsmönstret kan omvandlas till en riktig bild (2D) i en process som kallas bildrekonstruktion. Att det är möjligt att göra experiment som använder sig av diffraktion före destruktion med efterföljande bildrekonstruktion har visats för en mängd både icke-biologiska och biologiska prover i 2D. Nyligen har även den första 3D-modellen av ett biologiskt objekt rekonstruerats genom att kombinera många diffraktionsmönster som producerats med denna metod.

År 2008 visade forskare att det är teoretiskt möjligt att använda metoden för att avbilda små levande celler i 2D med subnanometerupplösning. Huvudarbetet av denna avhandling utgörs av en experimentell verifiering av denna förutsägelse. I Artikel I visar jag att genom att även med en XFEL som var betydligt svagare än den som användes i den teoretiskta studien kunde vi se diffraktion till en upplösning på 4 nm vilket motsvarar ca 40 atomdiametrar. Problemet med dessa bilder är att den starka signalen mättade detektorn. Vi var därför tvungna att använda svagare diffraktionsmönster för vidare analys. Detta reducerade upplösningen hos de rekonstruerade objekten. Vi har därefter föreslagit en förbättrad experimentuppställning som skulle undvika detektormättnad, men dessa har ännu inte testats experimentellt.

På grund av den inneboende variationen i storlek och form mellan olika celler är det långt ifrån trivialt att helt automatisera metoden för objektrekonstruktion från det registrerade diffraktionsmönstret, vilket skulle behövas för att i detalj kunna studera variabiliteten hos cellerna och deras molekylära maskiner. För att automatiken ska lyckas är det nödvändigt att kunna förutsäga formen på objektet från den uppmätta diffraktionen. Vidare måste även andra fenomen, såsom detektormättnad eller att flera celler befinner sig i strålen samtidigt kunna identifieras automatiskt.

För att lösa dessa problem har jag utvecklat programvaru-sviten RedFlamingo. RedFlamingo kan användas för att bedöma kvaliteten på enskilda bilder och producera ett antal viktiga parametrar som sen kan användas i bildrekonstruktionen. Om partikelns form är känd kan den användas för partikelurval (Artikel III). I vissa fall kan programmen också användas för att avgöra om den observerade variationen härrör från det biologiska provet, eller om det är en experimentell artefakt. Detta har visat sig vara användbart i processen att kombinera många 2D-bilder till en 3D-modell.

Jag är väldigt entusiastisk över att vara en del av utvecklingen av denna teknik. Mätning av diffraktion utan mättnadseffekter kan göra det möjligt att observera biologiska processer i levande celler. I framtiden ser jag studier av variabilitet hos isolerade molekylmaskiner som en första milstolpe. Därefter hoppas jag att dessa maskiner kan studeras tillsammans i sin naturliga miljö, och att vi därmed kan visualisera hur livet är organiserat på molekylär nivå.

\end{otherlanguage}