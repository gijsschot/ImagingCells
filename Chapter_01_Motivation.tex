%\part{Motivation}
\chapter{Motivation}
Cellular life and the organization of its constituents are amazingly intricate and diverse. Proteins form an interconnected and dynamic network in which specific changes to individual proteins can trigger a variety of global responses. In order to understand the factors that activate or deactivate various pathways we need to study  the entire living system, including individual components and their interactions with each other. A grand challenge of the 21st century is the imaging of live cells, at or near atomic resolution, with a time resolution that allows capturing the fastest biological processes. 

Super-resolution optical microscopy can image labeled parts of cells and has increased our understanding of cellular organization significantly \cite{Betzig2006}. However the technique requires the introduction of a fluorescent label, and it is ultimately limited by the size of this label. Nuclear Magnetic Resonance (NMR) can study the dynamics of proteins at atomic resolution, with picosecond time resolution \cite{Sapienza2010}. Recently it has succeeded in studying labelled proteins \textit{in vivo} \cite{Ye2013}. In general, NMR is limited by the size of proteins. I have participated in computationally predicting protein structure from backbone chemical shift only as shown in \textbf{Papers IV-IX}. A promising method to study cells as well as its constituents is electron cryo-microscopy. Focussed ion beams can be used to slice cryo-frozen cells into thin sections \cite{Marko2007}. This allowed the study of internal features of the cells. Using sub-tomogram averaging, the structure of highly abundant proteins can be elucidated at resolutions beyond 4 \AA  \cite{Schur2016}. For understanding rare events, however, acquisition time is a limiting factor. Although these results are incredible, cryo-electron microscopy does not study living cells. A further alternative that can be utilized to study living cells is femtosecond x-ray diffractive imaging (FXI). FXI uses ultra-short and extremely bright X-ray pulses produced by X-ray free-electron lasers (XFEL). The power of the pulse enables the measurement of interpretable signal from single bioparticles that otherwise would not scatter strong enough. The femtosecond pulse can outrun key damage processes in the sample. It is predicted that sub-nanometer resolution can be achieved on micron-sized cells with this method \cite{Bergh2008}. The femtosecond pulse gives an unprecedented time resolution that can capture the fastest biologically relevant motion at room temperature. Another big advantage of this method is the extremely high repetition rate. The recently operational European XFEL (EuXFEL) has a repetition rate of 27 000 Hz \cite{Altarelli2006}, potentially allowing over a billion images of a billion cells to be recorded in one day, and may open up new avenues of research in cell biology. 

This thesis deals with the experimental verification of FXI on living cells, and studies if, and what, computational and experimental tools are necessary to make high-resolution cell imaging a reality. This thesis will start by describing the general framework of the experiment: the interaction of light and matter, the generation of X-rays, the substrate-free sample delivery method, the recording of two- dimensional (2D) diffraction patterns, the reconstruction of images, how 2D images might be combined to derive three-dimensional (3D) structural information, and finally how image classification might be useful in this process. The final chapters describe the results on imaging \textit{living} cells, and how automated image classification has contributed to pattern selection and reconstruction.