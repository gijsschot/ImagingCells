\part{Motivation}
\chapter{Motivation}
Cellular life and the organization of its constituents are amazingly intricate and diverse. Proteins form an interconnected and dynamic network in which specific changes to individual proteins can trigger a variety of global responses. In order to understand the factors that activate or deactivate various pathways it is not enough to study individual components by themselves. A grand challenge of the 21st century is the imaging of live cells, at or near atomic resolution, with a time resolution that allows capturing even the fastest biological processes. 

Super-resolution optical microscopy can image labeled parts of cells and has increased our understanding of cellular organization significantly. However the technique requires the introduction of a fluorescent label, and it is ultimately limited by the size of this label. Nuclear Magnetic Resonance (NMR) can study the dynamics of proteins at atomic resolution, with nanosecond time resolution. Recently it has succeeded in studying labeled proteins \textit{in vivo} []. In general, NMR is limited by the size of proteins. I have participated in computationally predicting protein structure from backbone chemical shift only as shown in Papers VI-XI. A promising method to study cells as well as its constituents is electron cryo-microscopy (cryo-EM). Focussed ion beams can be used to slice cryo-frozen cells into thin sections. This has solved the issue of penetration depth allowing the studying of internal features. Using sub-tomogram averaging, the structure of highly abundant proteins can be elucidated at resolutions beyond 4A []. For understanding rare events, however, acquisition time is a limiting factor. The time resolution of this field is currently limited to the microsecond range []. Although these results are incredible, cryo-EM does not study living cells. A further alternative that can be utilized to study living cells is femtosecond x-ray diffractive imaging (FXI). FXI uses ultra-short and extremely bright pulses produced from x-ray free-electron lasers. The power of the pulse enables the measurement of interpretable signal from single bioparticles that otherwise would scatter strong enough. The femtosecond pulse can outrun key damage processes in the sample. It is predicted that sub-nanometer resolution can be achieved on micron-sized cells with this method [12]. The femtosecond pulse gives an unprecedented time resolution that captures any biologically relevant motion, all at room temperature. Another big advantage of this method is the extremely high repetition rate. The recently operational Eu- XFEL has a repetition rate of 27 000 Hz, potentially allowing the recording of over a billion images of a billion cells a day, and may open up new avenues of research in cell biology. 

This thesis deals with the experimental verification of FXI on living cells, and studies if, and what, computational and experimental tools are necessary to make cell imaging a reality. It will start by explaining the general framework necessary for each step of the experiment: The generation of X-rays, sample introduction, the interaction of light and matter, the recording of two- dimensional (2D) diffraction patterns, the reconstruction of cell images, how 2D images might be combined to derive three-dimensional (3D) structural information, and finally how image classification might be useful for the latter two. The final chapters describe the results on live cell imaging, and how image classification has been used for pattern selection and reconstruction.




