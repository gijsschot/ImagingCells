Sammanfatting pa svenska

In this thesis we want to develop new methods to image biological diversity at the molecular level. Understanding this diversity is very important as it often plays a major role in deceases. The thesis focuses on two topics: the imaging \textit{living} cells and the development of a computational tool to assess variability within a population of images.


In molecular biology structure and function are closely related. This however does not mean that each object has one single static shape, as for example a key has. On the contrary, many objects have multiple conformations and only through understanding the variation, their function becomes interpretable. A further complication arises from the fact that the variation often is highly dependent on the exact environment the objects are in. Molecular machines might start operating differently if they are operating outside of their natural environment. A tool that can images molecular machinery inside living cells (this is their natural environment) is therefore required to truly understand the structure of molecular life.


The amount of detail within objects (resolution) that one can image is limited by the wavelength of the light. Molecular biology is build up out of atoms, so light with a wavelength of the size of an atom is needed see single atoms and thus fully image molecular machines. This type of light is X-ray radiation. Another fundamental principle is that the smaller an object is, the stronger the beam of light that shines on the object needs to be, in order to record enough information to make an image. Strong beams have the disadvantage that they damage the objects. This led to the limitation that objects smaller than 100 atoms in diameter could not be visualized.


The development of a new type of X-ray laser (X-ray free-electron laser) that produces ultra-short and extremely brilliant X-ray pulses created the solution to the damage related resolution limit. The power of the pulse ensures that even individual molecular machines can be imaged in atomic detail. These pulses do damage the objects they image, but because they are ultra-short the damage only happens after the pulse. The recorded image is therefore of the undamaged object. This principle is called diffract-before-destroy. 

The word diffract indicates that we are not recording images of an object in the way a photo camera captures images. Instead we measure a so called diffraction pattern. The diffraction pattern can be converted to a real image in a process called image reconstruction. The feasibility of diffract-before-destroy and the image reconstruction process has been shown for a variety of non-biological and biological samples in 2D.  And recently the first 3D model of a biological object has been generated from many diffraction patterns generated using this method.

In 2008 researchers showed that it is theoretically possible to image small \textit{living} cells at sub-nanometer resolution in 2D. The main work of the thesis was an experimental test of this prediction. This study showed that, using a relatively weak XFEL compared to the predicted parameters, it is possible to record images upto 40 atoms resolution. The problem of these images is saturation. This posed a limit to resolution of the recovered object. We have suggested improvements to avoid detector saturation, but the suggestion have not been tested experimentally.


Due to the variability between different cells it is not trivial to fully automate the process of object recovery from the recorded image, something that is needed if one want to study the variability of cells, or the molecular machines inside them. For the automation to work it necessary to predict the shape of the object from the recorded image itself. Furthermore, other parameters such as saturation or having two cells in the beam at once also have to automatically assessed.


This led to the development of the software suite called RedFlamingo. Red Flamingo can be used to assess the quality of individual images, and deduces several essential features for image reconstruction. 


In some cases it can also determine whether the observed variety originate from a biological source, or whether it is a result from he experiment. It turns out that this feature might be useful in the process of deriving a 3D model from many 2D images.


I am very excited to see the development of this technique. The first focus will most likely be the studying of variability in isolated molecular machines, and hopefully after that these machines can be placed into their natural environment, allowing us to visualize the organisation of life.