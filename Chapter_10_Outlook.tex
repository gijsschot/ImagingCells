\chapter{Outlook}
There exists a conspicuous gap in knowledge about the organisation of life at the mesoscopic level. A technique that potentially can bridge this gap is FXI. The field of FXI is rapidly evolving and many of the initial technical limitation are being solved. For example tools for rapid feedback during experiments have been developed \cite{Daurer2016}, a new injection method has solved an outstanding problem in the delivery of particles smaller than 200 nm in diameter, and EMC is can now deal with conformational heterogeneity. With these developments I am very much looking forward to the first 3D models of a protein, possibly of more than one conformational state. 

This thesis showed progress on the other end of the gap. We show that we have not reached any fundamental limits for live cell imaging with an XFEL. We can introduce living cells in the XFEL beam, and record diffraction patterns at low signal to noise ratios, at high hit ratios. We recorded scattering beyond 4 nm resolution. With a more powerful X-ray laser sub-nanometer resolution on micron-sized cells should be possible. Image reconstruction has been successful for images that were not affected by detector saturation. The dynamic range of the detector can be extended with the help of a selective attenuator. We have build such a system, and are awaiting beam time. An increased dynamic range will almost certainly extend the resolution of future cell reconstructions. 

An automated reconstruction pipeline for heterogeneous objects is well on its way. The classification of cells on shape, intensity, as well as the exclusion of images affected by artefacts of the experimental procedure have been developed. The underlying algorithms have been combined in a computational framework called RedFlamingo.

With the rapid development of unsupervised machine learning algorithms and their extraordinary success in solving some specific questions, it would be very interesting to try to use their capabilities to for example try to, from an extensive set of reconstructions, find correlations between small variations in cell structure and for example certain external stimuli, or to general morphological changes such as cell division. One should not forget that in the past much has been learned from projection images, as, for example, most medical imaging ws two dimensional. Using the high repetition rates of XFELs, rare cellular events could be studied for the first time.

RedFlamingo has proven itself to be useful in a wide variety of experiments. As the data and the artefacts within it vary from experiment to experiment, different experiments often require different classification methods. The modular nature of RedFlamingo allowed for easy algorithm selection. It has been successful for pattern selection of small viruses of 70 nm to micrometer sized objects such as cells or giant viruses. RedFlamingo has proven itself also useful for pre-selecting patterns for EMC. This is important because too much sample heterogeneity will cause EMC to fail. Machine assisted pattern recognition might in the future also aid the process of image classification. 

To return to cell imaging, the most speculative and most interesting question of live cell imaging is: How do we obtain 3D information about the cell and its constituents. The addition of diffraction patterns from non-reproducible objects is very different from the addition of diffraction pattern from reproducible particles, as it results in a phenomenon called incoherent addition \cite{Maia2009}, which results in the loss of resolution. This makes it very difficult to retrieve 3D information from an heterogeneous object such as cells. 

A possible solutions to this problem might lie in a combination of multiple beams and sub-tomogram averaging. Right now it is practically impossible to build two XFELs at different angles. In the future this might be become more feasible as both accelerator and undulator hardware might become much smaller \cite{Tajima1979}. Moreover, not all beams have to be generated by XFELs, as one high resolution diffraction pattern might be aided by one or more low resolution diffraction patterns produced using a High Harmonic Generation X-ray source for example. The lower resolution diffraction pattern will aid in the elucidation of the third dimension. Using the extra information it might be possible to create a low resolution tomogram of the cell. The strength of the pulses and the number of beams will determine the resolution of the tomogram. One could then select reproducible parts of the tomogram, also known as sub-tomograms, combing them into a high resolution model of the constituent of that part. The subtomograms might even come from different cells. Key in this process is identification of the location of the reproducible parts. The high repetition rate and high temporal and spatial resolution obtainable using an XFEL might make it possible to collect enough data to visualize cellular events that occur infrequently, or are short lived. 