\message{ !name(SignalDetection.tex)}
\message{ !name(SignalDetection.tex) !offset(-2) }
\chapter{Data recording}
After the interaction between light and our material we have to
measure the scattered waves. Equation \ref{eq:full} shows that the
scattered wave is determined by both the amplitude of the wave and
its phase. Currently there is no device that can measure the phase of
x-rays directly, as it changes in the attosecond range. The former can be measured, but based on the low
probability between light and matter, as well as
the vacuum condition this is not trivial, and new detectors had to be
developed. In the satellite mission XMM-Newton of the European space
agency, two types of detectors were used. The CMOS and the pnCCD
type. In the experiments described in this thesis a pnCCD was used. 
\section{pnCCD}\label{sec:pnccd}
A pnCCD consists of three layers of material. First a layer
protecting the other layers from the high vacuum, second a layer
a layer of silicon and a third layer consisting of ... . Applying a
positive charge to the 
Silicon in a semi-conductor which means there is a energy gap between
a bound electron and an electron that can freely move through the material. The energy of x-rays is larger than this gap thus
through photo-ionisation electrons can be released. These electrons,
or electrons that are released in electron electron collisions
   
\section{Quantum Efficiency}
An important parameter describing the pnCCD is the quantum
efficiency. This parameter describes the number of incident photons
converted to charge carriers. Often this is close to 100\% but this
varies as a function of wavelength. High energy X-ray radiation would
require another type of detector than the pnCCD. 

\section{Missing Data}
The geometry of the detectors used in the experiments decribed in this
thesis consist of two moveable halves (5cm by 10 cm), from which a
semi circle with radius 1 cm is missing. The central
hole is created to let the high-intensity direct beam through. In this
area there is no intensity information present. 
 
\section{Saturation}
Besides missing data due to the detector geometry, detector saturation
can also be a problem. The potential wells described
in \ref{sec:pnccd} can maximally hold a certain current. Currents overflows with electrons.






\message{ !name(SignalDetection.tex) !offset(-48) }
