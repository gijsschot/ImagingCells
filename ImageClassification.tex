\chapter{Image Classification}
In order for algorithms such as EMC to work, the amount of heterogeneity within the diffraction data set has to be limited [TOMAS is there an article describing this?]. Furthermore, each of the steps in the experiment introduces it own type of noise to the measured diffraction pattern. For example think of the debris clumping around small particles compared to the drop when sprayed with the GDVN, detector malfunctioning or saturation, or intensity fluctuations due to the random start of the SASE process. Often the results of the first two types of noise can not be tolerated by EMC, and image classification before the EMC step is necessary. Also fast feedback about the first type of noise is very useful to have during the experiment. So far a very robust sizing method has been developed, but more extended methods might come in useful. Several methods shown here can give rapid feedback on the heterogeneity of the particle.

\section{Template-based classification}
If your object has a known shape it could be possible to only select the diffraction patterns that are similar to a set of expected diffraction patterns from the object called templates. Paper III explores the possibility of template-based classification. In general this method is highly dependent on the choice of template, as well as the amount and type of variation present in your sample.

\section{Feature extraction}
Another way of selecting diffraction patterns is by extracting general features from the diffraction pattern such as size, particle shape, amount of saturation, number of particles in the beam. Based on the relative values associated with the features patterns might be selected or not, or the experimental conditions might be rapidly adjusted. This section describes several methods to extract features.

\subsection{Size}
A common method to determine the size of an object is fitting the central speckle to the central speckle of simulated diffraction pattern from a sphere. This method has shown to be successful for particles that have an icosahedral shape to spherical shape \cite{Hantke2014,Daurer2017}. If the diffraction pattern also has the 3rd to the 5th minima present, the average of these minima can also be used to describe the size of the object. This method might be especially useful for strong scatterers, because the central speckle or first minimum might not be reliable due to saturation effects.

It might also be possible to determine the size and shape of an object from a filtered autocorrelation. This is useful when the object cannot be approximated by a sphere, but is rather elongated in one axis. Due to aerosol injection, the strongest contrast in the object is coming from the edge of the object. Therefore the edge is often one of the most clear features in an autocorrelation. This method is limited to convex objects.

\subsection{Multiple scatterers in the focus}


\subsection{Edge detection}
Some objects are characterized by having sharp edges. A sharp edge in real space corresponds to a streak in the direction perpendicular to the edge in Fourier space. Objects and/or orientations of objects might be classified by determining if streaks are present, and how many. Figure explains the idea behind a streak finding algorithm described in paper IV.