\part{Motivation}
\chapter{Motivation}
Cellular life and the organization of its constituents are amazingly intricate and diverse. Proteins form an interconnected network of interactions in which temporal changes to individual proteins can trigger global responses such as the reshaping of macrophages in the presence of a bacterium or a cancerous cell[]. In order to understand the factors that activate or deactivate such pathways it is not enough to study individual components by themselves. A grand challenge of the 21st century is the imaging of \textit{live} cells, at atomic resolution, at a time resolution that allows capturing even the fastest biological processes.

Super resolution optical microscopy can image cells as a whole and has increased our understanding of cellular organization significantly. However the technique is ultimately limited by the size of the fluorescent label. X-ray crystallography can image large bio-particles at atomic resolution, but requires the particles to crystallize. For cells this is not possible. Nuclear Magnetic Resonance can study the dynamics of proteins at atomic resolution, at with nanosecond time resolution. Recently it has succeeded in studying labelled proteins \textit{in vivo} []. NMR is still limited by the size of the proteins it can study, as well as the collection time. It has been attempted to predict protein structure from backbone chemical shift only, which would avoid the must time consuming steps [Shen]. I have participated in the improvement of this method as shown in Paper V-X. This method works well for predicting the structure of ordered region.

The most promising method to study cells as well as its constituents at near-atomic resolution is cryo-electron microscopy. Focussed ion beams can be used to slice cryo-frozen cells into thin sections without considerably damaging them. This has solved the issue of penetration debt allowing the studying of internal features. Using sub-tomogram averaging, the structure of highly abundant proteins can be elucidated at resolutions beyond 4A []. For understanding rare events, however, acquisition time is a limiting factor. The time resolution of this field is currently limited to the microsecond range [].

A promising alternative that can be utilized to study living cells is femtosecond x-ray diffraction imaging (FXI). FXI uses the ultrashort and extremely bright pulses produced by free-electron x-ray lasers. The power of the pulse allows for single particle imaging (SPI). They femtosecond pulse outrun the key damage processes that occur due to the exposure to the beam. It is predicted that sub-nanometer resolution can be achieved on micron-sized cells  \cite{Bergh2008}. The femtosecond pulse gives an unprecedented time resolution that captures any biologically relevant motion, all at room temperature. Another big advantage of this method is the repetition rate. The recently operational EuXFEL can have repetition rates of 27000 Hz, potentially allowing the recording of over a billion images a day. This might enable the imaging of rare cellular events.

This thesis deals with the experimental verification of FXI on living cells, and studies if, and what, computational and experimental tools are necessary to make cell imaging a reality. It will start by explaining the general framework necessary for each step of the experiment: The generation of X-rays, sample introduction, the interaction of light and matter, the recording of two-dimensional (2D) diffraction pattern, the reconstruction of cell images, how 2D images might be combined to derive three-dimensional (3D) structural information, and finally how image classification might be useful for the latter two. The final chapters describe the results on life cell imaging, and how image classification has been used for imaging reconstruction and pattern selection.



